\section{Introduction}
The outbreak of a new crown outbreak depends on many factors, among which human influences include population mobility patterns, mask prevalence, and vaccination rates. The research in our project addresses the three important factors mentioned above. All three factors are influenced to some extent by government policies, and government decisions can be seen to influence the trend of the epidemic. We selected two outbreak trends in Singapore, which are listed below:
\begin{itemize}
	\item Singapore, March 2020 to Sept 2020 (optional)
	\item Singapore, July 2021 till now, mainly delta variant
\end{itemize}

There are significant differences among these two outbreaks. For example, during the last year's outbreak in Singapore, there were no available vaccines, but the government had published very strict policies to reduce social interactions. And for this year's outbreak in Singapore, nearly 90\% of people have been fully vaccinated, but the government is entering the `Preparatory Stage' and the local restrictions have been relaxed. The difference in vaccination rates and population mobility can help us to make a more accurate analysis of the impact of vaccination, mask policies and mobility. We hope that by studying these three factors and analyzing their weights on the outbreak trends, we can help the government to develop more effective prevention strategies to achieve successful protests. At the same time, we verify whether our research model is reasonable through data to achieve the purpose of accurately predicting the future trend of the epidemic.