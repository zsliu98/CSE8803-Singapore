\section{Literature review}
We summarize and outline some information from papers related to agent models, including a new approach to predict the spread of COVID-19 in facilities based on agent models, the COVID-ABS agent model, the impact of the number of masks used globally on the outbreak discussed based on agent models, and the analysis of Australian data based on agent models. One of the emerging approaches based on agent models is used to represent complex systems comprising agents whose behavior is specified using simple rules \cite{cuevas2020agent}. This method differs from mathematical analyses of LGBT communities in that it represents individuals with a variety of characteristics, resulting in more realistic results. An agent-based strategy is offered for assessing COVID-19 transmission concerns in facilities. The spatio-temporal transmission mechanism is modeled in this way. Simulating spatio-temporal transmission mechanisms and simulating agents making judgments based on pre-programmed criteria are the foundations of this technology. These rules represent the geographical patterns and infectious situations with which the agents interact to explain the infection process. The model is extremely versatile, allowing for the testing of various hypotheses as well as alternative scenarios by considering hypothetical conditions that cannot be studied in real life. When compared to experimental methods, using this agent-based model offers the advantage of saving time and money. The ABS proxy model \cite{silva2020covid}, which simulates the epidemiological and economic impact of the COVID-19 pandemic in closed societies and whose results can be generalized in a broader context and used by government rulers to predict social policies and evaluate their effectiveness in real-world scenarios, is similar to this. This model creates novel situations by taking into account the unique characteristics of several research areas. The future study hopes to improve the model by incorporating methods for closing and opening organizations, as well as the ability to fire workers. Based on various scenarios as well as government planning, the model can potentially be optimized as a library. In another paper \cite{kai2020universal}, the movement of ABS agent models to predict the effect of masking on virus transmission is also mentioned in the literature. The paper compared two different pandemic viruses, including the recent COVID-19, based on this agent-based model, and the researchers compared the model's predictions with a large body of data to show a strong correlation between mask-wearing and the increase and decrease in daily cases. Fine-grained computational simulations \cite{chang2020modelling} of the pandemic were generated for Australia based on an agent model, and strategies to control for COVID-19 mitigation and suppression were introduced from the simulation data results, which were calibrated to match key features of COVID-19 transmission, with an important calibration result being the age-related scores of symptomatic cases. Above we have discussed and referred to some new technologies about the application of the proxy model and the impact of using this model on the trend of wearing masks on the pandemic and the prediction application based on Australian data and proxy models. These favorable information and analysis are useful for our project. Progress has helped a lot and brought us more ideas.